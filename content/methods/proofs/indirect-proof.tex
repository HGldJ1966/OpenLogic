% Part: methods
% Chapter: proofs
% Section: indirect-proof

\documentclass[../../../include/open-logic-section]{subfiles}

\begin{document}

\olfileid{mth}{prf}{ind}

\olsection{Indirect Proof} 

In the first instance, indirect proof is an inference pattern that is
used to prove negative claims.  Suppose you want to show that some
claim~$p$ is \emph{false}, i.e., you want to show~$\lnot p$.  A
promising strategy---and in many cases the only promising
strategy---is to (a) suppose that $p$~is true, and (b) show that this
assumption leads to something you know to be false.  ``Something known
to be false'' may be a result that conflicts with---contradicts---$p$
itself, or some other hypothesis of the overall claim you are
considering.  For instance, a proof of ``if $q$ then $\lnot p$''
involves assuming that $q$~is true and proving~$\lnot p$ from it. If
you prove $\lnot p$ indirectly, that means assuming $p$ in addition
to~$q$. If you can prove $\lnot q$ from $p$, you have shown that the
assumption~$p$ leads to something that contradicts your other
assumption~$q$, since $q$~and $\lnot q$ cannot both be true.
Therefore, indirect proofs are also often called ``proofs by
contradiction.'' Of course, you have to use other inference patterns
in your proof of the contradiction, as well as unpacking definitions.
Let's consider an example.

\begin{prop}
  If $X \subseteq Y$ and $Y = \emptyset$, then $X = \emptyset$.
\end{prop}

\begin{proof}
Since this is a conditional claim, we assume the antecedent and want
to prove the consequent:
\begin{quote}
Suppose $X \subseteq Y$ and $Y = \emptyset$. We want to show that $X =
\emptyset$.
\end{quote}
Now let's consider the definition of $\emptyset$ and $=$ for sets. $X
= \emptyset$ if every !!{element} of~$X$ is also an !!{element}
of~$\emptyset$ and (vice versa). And $\emptyset$ is defined as the set
with no !!{element}s. So $X = \emptyset$ iff $X$ has no !!{element}s,
i.e., it's not the case that there is an~$x \in X$.
\begin{quote}
$X = \emptyset$ iff there is no $x \in X$.
\end{quote}
So we've determined that what we want to prove is really a negative
claim $\lnot p$, namely: it's not the case that there is an $x \in X$.
To use indirect proof, we have to assume the corresponding positive
claim~$p$, i.e., there is an $x \in X$.  We indicate that we're doing
an indirect proof by writing ``We proceed indirectly:'' or, ``By way
of contradiction,'' or even just ``Suppose not.'' We then state the
assumption of~$p$.
\begin{quote}
We proceed indirectly. Suppose there is an $x \in X$.
\end{quote}
This is now the new assumption we'll use to obtain a contradiction. We
have two more assumptions: that $X \subseteq Y$ and that $Y =
\emptyset$. The first gives us that $x \in Y$:
\begin{quote}
Since $X \subseteq Y$, $x \in Y$.
\end{quote}
But now by unpacking the definition of $Y = \emptyset$ as before, we
see that this conclusion conflicts with the second assumption. Since
$x \in Y$ but $x \notin \emptyset$, we have $Y \neq \emptyset$.
\begin{quote}
Since $x \in Y$ but $x \notin \emptyset$, $Y \neq \emptyset$. This
contradicts the assumption that~$Y = \emptyset$.
\end{quote}
This already completes the proof: we've arrived at what we need (a
contradiction) from the assumptions we've set up, and this means that
the assumptions can't all be true. Since the first two assumptions ($X
\subseteq Y$ and $Y = \emptyset$) are not contested, it must be the
last assumption introduced (there is an $x \in X$) that must be
false. But if we want to be through, we can spell this out.
\begin{quote}
Thus, our assumption that there is an $x \in X$ must be false, hence,
$X = \emptyset$ by indirect proof. \qedhere
\end{quote}
\end{proof}

Every positive claim is trivially equivalent to a negative claim: $p$
iff $\lnot\lnot p$.  So indirect proofs can also be used to establish
positive claims: To prove $p$, read it as the negative claim
$\lnot\lnot p$. If we can prove a contradiction from $\lnot p$, we've
established $\lnot\lnot p$ by indirect proof, and
hence~$p$. Crucially, it is sometimes easier to work with $\lnot p$ as
an assumption than it is to prove~$p$ directly.  And even when a
direct proof is just as simple (as in the next example), some people
prefer to proceed indirectly.  If the double negation confuses you,
think of an indirect proof of some claim as a proof of a contradiction
from the \emph{opposite} claim. So, an indirect proof of $\lnot p$ is a proof
of a contradiction from the assumption~$p$; and indirect proof of~$p$
is a proof of a contradiction from~$\lnot p$.

\begin{prop}
$X \subseteq X \cup Y$.
\end{prop}

\begin{proof}
On the face of it, this is a positive claim: every $x \in X$ is also
in $x \cup Y$.  The opposite of that is: some $x \in X$ is $\notin X
\cup Y$. So we can prove it indirectly by assuming this opposite
claim, and showing that it leads to a contradiction.
\begin{quote}
  Suppose not, i.e., $X \nsubseteq X \cup Y$.
\end{quote}
We have a definition of $X \subseteq X \cup Y$: every $x \in X$ is
also $\in X \cup Y$.  To understand what $X \nsubseteq X \cup Y$
means, we have to use some elementary logical manipulation on the
unpacked definition: it's false that every $x \in X$ is also $\in X
\cup Y$ iff there is \emph{some}~$x \in X$ that is $\notin Z$.  (This
is a place where you want to be very careful: many students' attempted
indirect proofs fail because they analyze the negation of a claim like
``all $A$s are $B$s'' incorrectly.) In other words, $X \nsubseteq X
\cup Y$ iff there is an $x$ such that $x \in X$ and $x \notin X \cup
Y$. From then on, it's easy.
\begin{quote}
So, there is an $x \in X$ such that $x \notin X \cup Y$.  By
definition of $\cup$, $x \in X \cup Y$ iff $x \in X$ or $x \in
Y$. Since $x \in X$, we have $x \in X \cup Y$. This contradicts the
assumption that $x \notin X \cup Y$. \qedhere
\end{quote}
\end{proof}

\begin{prob}
Prove \emph{indirectly} that $X \cap Y \subseteq X$.
\end{prob}

\begin{prop}
If $X \subseteq Y$ and $Y \subseteq Z$ then $X \subseteq Z$.
\end{prop}

\begin{proof}
First, set up the required conditional proof:
\begin{quote}
Suppose $X \subseteq Y$ and $Y \subseteq Z$. We want to show $X
\subseteq Z$.
\end{quote}
Let's proceed indirectly.
\begin{quote}
Suppose not, i.e., $X \nsubseteq Z$.
\end{quote}
As before, we reason that $X \nsubseteq Z$ iff not every $x \in X$ is
also $\in Z$, i.e., some $x \in X$ is $\notin Z$.  Don't worry, with
practice you won't have to think hard anymore to unpack negations like
this.
\begin{quote}
In other words, there is an~$x$ such that $x \in X$ and $x \notin Z$.
\end{quote}
Now we can use the assumption that (some) $x \in X$ and $x \notin Z$
to get to our contradiction. Of course, we'll have to use the other
two assumptions to do it.
\begin{quote}
Since $X \subseteq Y$, $x \in Y$. Since $Y \subseteq Z$, $x \in
Z$. But this contradicts $x \notin Z$. \qedhere
\end{quote}
\end{proof}

\begin{prop}
If $X \cup Y = X \cap Y$ then $X = Y$.
\end{prop}

\begin{proof}
The beginning is now routine:
\begin{quote}
  Suppose $X \cup Y = X \cap Y$. Assume, by way of contradiction, that
  $X \neq Y$.
\end{quote}
Our assumption for the indirect proof is that $X \neq Y$. Since $X =
Y$ iff $X \subseteq Y$ an $Y \subseteq X$, we get that $X \neq Y$ iff
$X \nsubseteq Y$ \emph{or} $Y \nsubseteq X$. (Note how important it is
to be careful when manipulating negations!{}) To prove a contradiction
from this disjunction, we use a proof by cases and show that in each
case, a contradiction follows.
\begin{quote}
$X \neq Y$ iff $X \nsubseteq Y$ or $Y \nsubseteq X$. We distinguish cases.
\end{quote}
In the first case, we assume $X \nsubseteq Y$, i.e., for some $x$, $x
\in X$ but $\notin Y$. $X \cap Y$ is defined as those !!{element}s
that $X$ and $Y$ have in common, so if something isn't in one of them
it's not in the intersection. $X \cup Y$ is $X$ together with $Y$, so
anything in either is also in the union. This tells us that $x \in X
\cup Y$ but $x \notin X \cap Y$, and hence that $X \cap Y \neq Y \cap
X$.
\begin{quote}
  Case 1: $X \nsubseteq Y$. Then for some $x$, $x \in X$ but $x \notin
  Y$. Since $x \notin Y$, then $x \notin X \cap Y$. Since $x \in X$,
  $x \in X \cup Y$. So, $X \cap Y \neq Y \cap X$, contradicting the
  assumption that $X \cap Y = X \cup Y$.

  Case 2: $Y \nsubseteq X$. Then for some $y$, $y \in Y$ but $y \notin
  X$. As before, we have $y \in X \cup Y$ but $y \notin X \cap Y$, and
  so $X \cap Y \neq X \cup Y$, again contradicting $X \cap Y = X \cup
  Y$. \qedhere
\end{quote}
\end{proof}

\end{document}
