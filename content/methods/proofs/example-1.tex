% Part: methods
% Chapter: proofs
% Section: example-1

\documentclass[../../../include/open-logic-section]{subfiles}

\begin{document}

\olfileid{mth}{prf}{ex1}

\olsection{An Example} 

Our first example is the following simple fact about unions and
intersections of sets.  It will illustrate unpacking definitions,
proofs of conjunctions, of universal claims, and proof by cases.

\begin{prop}
For any sets $X$, $Y$, and $Z$, $X \cup (Y \cap Z) = (X \cup Y)
\cap (X \cup Z)$
\end{prop}

Let's prove it!{}

\begin{proof}

First we unpack the definition of ``$=$'' in the statement of the
proposition. Recall that proving equality between sets means showing
that the sets have the same elements. That is, all elements of $X \cup
(Y \cap Z)$ are also elements of $(X \cup Y) \cap (X \cup Z)$, and
vice versa.  The ``vice versa'' means that also every element of $(X
\cup Y) \cap (X \cup Z)$ must be an !!{element} of $X \cup (Y \cap
Z)$.  So in unpacking the definition, we see that we have to prove a
conjunction.  Let's record this:

\begin{quote}
By definition, $X \cup (Y \cap Z) = (X \cup Y) \cap (X \cup Z)$ iff
every !!{element} of $X \cup (Y \cap Z)$ is also !!a{element} of $(X
\cup Y) \cap (X \cup Z)$, and every !!{element} of $(X \cup Y) \cap (X
\cup Z)$ is !!a{element} of $X \cup (Y \cap Z)$.
\end{quote}

Since this is a conjunction, we must prove each conjunct
separately. Lets start with the first: let's prove that every
!!{element} of $X \cup (Y \cap Z)$ is also !!a{element} of $(X
\cup Y) \cap (X \cup Z)$.

This is a universal claim, and so we consider an arbitrary !!{element}
of $X \cup (Y \cap Z)$ and show that it must also be !!a{element} of
$(X \cup Y) \cap (X \cup Z)$. We'll pick a variable to call this
arbitrary !!{element} by, say,~$z$.  Our proof continues:

\begin{quote}
First, we prove that every !!{element} of $X \cup (Y \cap Z)$ is also
!!a{element} of $(X \cup Y) \cap (X \cup Z)$. Let $z \in X \cup (Y
\cap Z)$. We have to show that $z \in (X \cup Y) \cap (X \cup Z)$.
\end{quote}  

Now it is time to unpack the definition of $\cup$ and~$\cap$. For
instance, the definition of $\cup$ is: $X \cup Y = \Setabs{z}{z \in X
  \text{ or } z \in Y}$.  When we apply the definition to ``$X \cup (Y
\cap Z)$,'' the role of the ``$Y$'' in the definition is now played by
``$Y \cap Z$,'' so $X \cup (Y \cap Z) = \Setabs{z}{z \in X \text{ or }
  z \in Y \cap Z}$.  So our assumption that $z \in X \cup (Y \cap Z)$
amounts to: $z \in \Setabs{z}{z \in X \text{ or } z \in Y \cap
  Z}$. And $z \in \Setabs{z}{\dots z\dots}$ iff \dots $z$ \dots, i.e.,
in this case, $z \in X$ or $z \in Y \cap Z$.

\begin{quote}
By the definition of $\cup$, either $z \in X$ or $z \in Y \cap Z$.
\end{quote}

Since this is a disjunction, it will be useful to apply proof by
cases. So we take the two cases, and show that in each one, the
conclusion we're aiming for (namely, ``$z \in (X \cup Y) \cap (X \cup
Z)$) obtains.

\begin{quote}
Case 1: Suppose that $z \in X$.
\end{quote}

There's not much more to work from based on our assumptions. So let's
look at what we have to work with in the conclusion. We want to show
that $z \in (X \cup Y) \cap (X \cup Z)$. Based on the definition of
$\cap$, if we want to show that $z \in (X \cup Y) \cap (X \cup Z)$, we
have to show that it's in both $(X \cup Y)$ and $(X \cup Z)$. The
answer is immediate. But $z \in X \cup Y$ iff $z \in X$ or $z \in Y$,
and we already have (as the assumption of case~1) that $z \in X$. By
the same reasoning---switching $Z$ for $Y$---$z \in X \cup Z$. This
argument went in the reverse direction, so let's record our reasoning
in the direction needed in our proof.

\begin{quote}
Since $z \in X$, $z \in X$ or $z \in Y$, and hence, by definition
of~$\cup$, $z \in X \cup Y$. Similarly, $z \in X \cup Z$.  But this
means that $z \in (X \cup Y) \cap (X \cup Z)$, by definition
of~$\cap$.
\end{quote}

This completes the first case of the proof by cases. Now we want to
derive the conclusion in the second case, where $z \in Y \cap Z$.

\begin{quote}
Case 2: Suppose that $z \in Y \cap Z$.
\end{quote}

Again, we are working with the intersection of two sets. Since $z \in
Y \cap Z$, $z$ must be an element of both $Y$ and $Z$.

\begin{quote}
Since $z \in Y \cap Z$, $z$ must be an element of both $Y$ and $Z$, by
definition of~$\cap$.
\end{quote}

It's time to look at our conclusion again. We have to show that $z$ is
in both $(X \cup Y)$ and $(X \cup Z)$. And again, the solution is
immediate.

\begin{quote}
Since $z \in Y$, $z \in (X \cup Y)$. Since $z \in Z$, also $z \in (X
\cup Z)$.  So, $z \in (X \cup Y) \cap (X \cup Z)$.
\end{quote}

Here we applied the definitions of $\cup$ and $\cap$ again, but since
we've already recalled those definitions, and already showed that if
$z$ is in one of two sets it is in their union, we don't have to be as
explicit in what we've done.

We've completed the second case of the proof by cases, so now we can
assert our first conclusion.

\begin{quote}
So, if $z \in X \cup (Y \cap Z)$ then $z \in (X \cup Y) \cap (X \cup Z)$.
\end{quote}

Now we just want to show the other direction, that every !!{element}
of $(X \cup Y) \cap (X \cup Z)$ is !!a{element} of $X \cup (Y \cap
Z)$. As before, we prove this universal claim by assuming we have an
arbitrary !!{element} of the first set and show it must be in the
second set. Let's state what we're about to do.

\begin{quote}
Now, assume that $z \in (X \cup Y) \cap (X \cup Z)$. We want to show
that $z \in X \cup (Y \cap Z)$.
\end{quote}

We are now working from the hypothesis that $z \in (X \cup Y) \cap (X
\cup Z)$. It hopefully isn't too confusing that we're using the
same~$z$ here as in the first part of the proof.  When we finished
that part, all the assumptions we've made there are no longer in
effect, so now we can make new assumptions about what $z$ is.  If that
is confusing to you, just replace $z$ with a different variable in
what follows.

We know that $z$ is in both $X \cup Y$ and $X \cup Z$, by definition
of~$\cap$. And by the definition of $\cup$, we can further unpack this
to: either $z \in X$ or $z \in Y$, and also either $z \in X$ or $z \in
Z$. This looks like a proof by cases again---except the ``and' makes
it confusing. You might think that this amounts to there being three
possibilities: $z$ is either in $X$, $Y$ or $Z$. But that would be a
mistake.  We have to be careful, so let's consider each disjunction in
turn.

\begin{quote}
By definition of $\cap$, $z \in X \cup Y$ and $z \in X \cup Z$. By
definition of $\cup$, $z \in X$ or $z \in Y$. We distinguish cases.
\end{quote}

Since we're focusing on the first disjunction, we haven't unpacked the
second one yet. In fact, we don't need it. The first case is $z \in
X$, and !!a{element} of a set is also !!a{element} of that union of
that set with any other. So case~1 is easy:

\begin{quote}
Case 1: Suppose that $z \in X$. It follows that $z \in X \cup (Y \cap
Z)$.
\end{quote}

Now for the second case, $z \in Y$. Here we'll unpack the second
$\cup$ and do another proof-by-cases:

\begin{quote}
Case 2: Suppose that $z \in Y$. Since $z \in X \cup Z$, either $z \in
X$ or $z \in Z$. We distinguish cases further:

Case 2a: $z \in X$. Then, again, $z \in X \cup (Y \cap Z)$.
\end{quote}

Ok, this was a bit weird. We didn't actually need the assumption
that~$z \in Y$ for this case, but that's ok.

\begin{quote}
Case 2b: $z \in Z$. Then $z \in Y$ and $z \in Z$, so $z \in Y \cap Z$,
and consequently, $z \in X \cup (Y \cap Z)$.
\end{quote}

This concludes both proof-by-cases and so we're done with the second
half. Since we've proved both directions, the proof is complete.

\begin{quote}
So, if $z \in (X \cup Y) \cap (X \cup Z)$ then $z \in X \cup (Y \cap Z)$.
  
Together, we've showed that $X \cup (Y \cap Z) = (X \cup Y) \cap (X
\cup Z)$.
\end{quote}
\end{proof}

\end{document}
