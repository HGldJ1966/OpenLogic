% Part: first-order-logic
% Chapter: natural-deduction
% Section: provability

% verification of properties of provability needed for maximally
% consistent sets in the completeness chapter.

\documentclass[../../../include/open-logic-section]{subfiles}

\begin{document}

\olfileid{fol}{ntd}{prv}
\olsection{Properties of \usetoken{S}{derivability}}

We will now establish a number of properties of the !!{derivability}
relation.  They are independently interesting, but each will play a
role in the proof of the completeness theorem.

\begin{prop}[Monotony]
\ollabel{prop:monotony}
If $\Gamma \subseteq \Delta$ and $\Gamma \Proves !A$, then $\Delta
\Proves !A$.
\end{prop}

\begin{proof}
Any !!{derivation} of $!A$ from $\Gamma$ is also !!a{derivation} of
$!A$ from~$\Delta$.
\end{proof}

\begin{prop}[Compactness]
  \ollabel{prop:proves-compact}
  \begin{enumerate}
  \item If $\Gamma \Proves !A$ then there is a finite subset $\Gamma_0
    \subseteq \Gamma$ such that $\Gamma_0 \Proves !A$.
  \item If $\Gamma$ is consistent, every finite subset of~$\Gamma$ is
    consistent.
  \end{enumerate}
\end{prop}

\begin{proof}
  \begin{enumerate}
    \item If $\Gamma \Proves !A$, then there is !!a{derivation}
      of~$!A$ from~$\Gamma$. But any !!{derivation} is finite, so can
      only contain finitely many !!{undischarged} assumptions. Let
      $\Gamma_0$ be the set of !!{undischarged} assumptions of the
      !!{derivation}; it is !!a{derivation} of~$!A$ from a
      finite~$\Gamma_0 \subseteq \Gamma$.
    \item Follows from (1) by taking $!A \ident \lfalse$.
  \end{enumerate}
\end{proof}

\begin{prop}
\ollabel{prop:provability-contr} If $\Gamma \Proves
  !A$ and $\Gamma \cup \{!A\}$ is inconsistent, then
  $\Gamma$ is inconsistent.
\end{prop}

\begin{proof}
Let the !!{derivation} of $!A$ from $\Gamma$ be
  $\delta_1$ and the !!{derivation} of $\lfalse$ from $\Gamma \cup \{!A\}$
  be $\delta_2$. We can then !!{derive}:
\[
\AxiomC{$\Discharge{!A}{1}$}
\RightLabel{$\delta_2$}
\DeduceC{$\lfalse$}
\DischargeRule{\Intro{\lnot}}{1}
\UnaryInfC{$\lnot !A$}
\AxiomC{}
\RightLabel{$\delta_1$}
\DeduceC{$!A$}
\RightLabel{\Elim{\lnot}}
\BinaryInfC{$\lfalse$}
\DisplayProof
\]
In the new !!{derivation}, the assumption $!A$ is !!{discharged}, so it is
!!a{derivation} from $\Gamma$.
\end{proof}

\begin{prop}
\ollabel{prop:provability-lnot} If $\Gamma \cup \{!A\}$ is
inconsistent, then $\Gamma \Proves \lnot !A$.
\end{prop}

\begin{proof}
Suppose that $\Gamma \cup \{!A\}$ is inconsistent. Then there is
!!a{derivation} of $\lfalse$ from $\Gamma \cup \{!A\}$.  Let $\delta$
be the !!{derivation} of $\lfalse$, and consider
\[
\AxiomC{$\Discharge{!A}{1}$}
\RightLabel{$\delta$}
\DeduceC{$\lfalse$}
\DischargeRule{\Intro{\lnot}}{1}
\UnaryInfC{$\lnot !A$}
\DisplayProof
\]
\end{proof}

\begin{prop}
  \ollabel{prop:explicit-inc} If $\Gamma \Proves !A$ and
  $\Gamma \Proves \lnot !A$, then $\Gamma$ is inconsistent.
\end{prop}

\begin{proof}
  This is a simple application of the $\Elim{\lnot}$ rule.
\end{proof}


\begin{prop}
\ollabel{prop:provability-exhaustive} If $\Gamma \cup \{!A\}$ is
inconsistent and $\Gamma \cup \{\lnot !A\}$ is inconsistent, then
$\Gamma$ is inconsistent.
\end{prop}

\begin{proof}
There are !!{derivation}s $\delta_1$ and $\delta_2$ of $\lfalse$ from
  $\Gamma \cup \{ !A \}$ and $\lfalse$ from $\Gamma \cup \{ \lnot !A
  \}$, respectively. We can then !!{derive}
\[
\AxiomC{$\Discharge{\lnot !A}{2}$}
\RightLabel{$\delta_2$}
\DeduceC{$\lfalse$}
\DischargeRule{\Intro{\lnot}}{2}
\UnaryInfC{$\lnot \lnot !A$}
\AxiomC{$\Discharge{!A}{1}$}
\RightLabel{$\delta_1$}
\DeduceC{$\lfalse$}
\DischargeRule{\Intro{\lnot}}{1}
\UnaryInfC{$\lnot !A$}
\RightLabel{\Elim{\lnot}}
\BinaryInfC{$\lfalse$}
\DisplayProof
\]
Since the assumptions $!A$ and $\lnot !A$ are !!{discharged}, this is
!!a{derivation} of~$\lfalse$ from~$\Gamma$ alone. Hence $\Gamma$ is
inconsistent.
\end{proof}

\begin{prop}
\ollabel{prop:provability-lor-left} If $\Gamma \cup \{!A\}$ and and
$\Gamma \cup \{!B\}$ are both inconsistent, then $\Gamma \cup \{!A
\lor !B\}$ is inconsistent.
\end{prop}

\begin{proof}
  Exercise.
\end{proof}

\begin{prob}
Prove \olref[fol][ntd][prv]{prop:provability-lor-left}
\end{prob}
  
\begin{prop}
\ollabel{prop:provability-lor-right} If $\Gamma \Proves !A$ or $\Gamma
\Proves !B$, then $\Gamma \Proves !A \lor !B$.
\end{prop}

\begin{proof}
\iftag{probOr}{Exercise.}{%
Suppose $\Gamma \Proves !A$. There is !!a{derivation} $\delta$ of $!A$
from~$\Gamma$. We can !!{derive}
\[
\AxiomC{}
\RightLabel{$\delta$}
\DeduceC{$!A$}
\RightLabel{\Intro{\lor}}
\UnaryInfC{$!A \lor !B$}
\DisplayProof
\]
Therefore $\Gamma \Proves !A \lor !B$. The proof for when $\Gamma
\Proves !B$ is similar.}
\end{proof}

\begin{probtag}{probOr}
Prove \olref[fol][ntd][prv]{prop:provability-lor-right}.
\end{probtag}
  
\begin{prop}
\ollabel{prop:provability-land-left} If $\Gamma \Proves !A \land !B$
then $\Gamma \Proves !A$ and $\Gamma \Proves !B$.
\end{prop}

\begin{proof}
\iftag{probAnd}{Exercise}{If $\Gamma \Proves !A \land !B$, there is
  !!a{derivation} $\delta$ of $!A \land !B$ from $\Gamma$. Consider
\[
\AxiomC{}
\RightLabel{$\delta$}
\DeduceC{$!A \land !B$}
\RightLabel{\Elim{\land}}
\UnaryInfC{$!A$}
\DisplayProof
\]
Hence, $\Gamma \Proves !A$.  A similar !!{derivation} 
shows that $\Gamma \Proves !B$.}
\end{proof}

\begin{probtag}{probAnd}
Prove \olref[fol][ntd][prv]{prop:provability-land-left}.
\end{probtag}

\begin{prop}
\ollabel{prop:provability-land-right} If $\Gamma \Proves !A$ and
$\Gamma \Proves !B$, then $\Gamma \Proves !A \land !B$.
\end{prop}

\begin{proof}
\iftag{probAnd}{Exercise.}{If $\Gamma \Proves !A$ as well as $\Gamma
  \Proves !B$, there are !!{derivation}s $\delta_1$ of $!A$ and
  $\delta_2$ of $!B$ from $\Gamma$.  Consider
\[
\AxiomC{}
\RightLabel{$\delta_1$}
\DeduceC{$!A$}
\AxiomC{}
\RightLabel{$\delta_2$}
\DeduceC{$!B$}
\RightLabel{\Intro{\land}}
\BinaryInfC{$!A \land !B$}
\DisplayProof
\]
The undischarged assumptions of the new !!{derivation} are all
in~$\Gamma$, so we have $\Gamma \Proves !A \land !B$.}
\end{proof}

\begin{probtag}{probAnd}
Prove \olref[fol][ntd][prv]{prop:provability-land-right}.
\end{probtag}

\begin{prop}
\ollabel{prop:provability-mp} If $\Gamma \Proves !A$ and $\Gamma
\Proves !A \lif !B$, then $\Gamma \Proves !B$.
\end{prop}

\begin{proof}
\iftag{probIf}{Exercise.}{Suppose that $\Gamma \Proves !A$ and $\Gamma
  \Proves !A \lif !B$.  There are !!{derivation}s $\delta_1$ of $!A$
  from $\Gamma$ and $\delta_2$ of $!A \lif !B$ from
  $\Gamma$. Consider:
\[
\AxiomC{}
\RightLabel{$\delta_2$}
\DeduceC{$!A \lif !B$}
\AxiomC{}
\RightLabel{$\delta_1$}
\DeduceC{$!A$}
\RightLabel{\Elim{\lif}}
\BinaryInfC{$!B$}
\DisplayProof
\]
This means that $\Gamma \Proves !B$.}
\end{proof}

\begin{probtag}{probIf}
Prove \olref[fol][ntd][prv]{prop:provability-mp}.
\end{probtag}

\begin{prop}
\ollabel{prop:provability-lif} If $\Gamma \Proves \lnot !A$ or $\Gamma
\Proves !B$, then $\Gamma \Proves !A \lif !B$.
\end{prop}

\begin{proof}
\iftag{probIf}{Exercise.}{First suppose $\Gamma \Proves \lnot !A$.
  Then there is a !!{derivation}~$\delta$ of $\lnot !A$ from~$\Gamma$.  The
  following !!{derivation} shows that $\Gamma \Proves !A \lif !B$:
\[
\AxiomC{}
\RightLabel{$\delta$}
\DeduceC{$\lnot !A$}
\AxiomC{$\Discharge{!A}{1}$}
\RightLabel{\Elim{\lnot}}
\BinaryInfC{$\lfalse$}
\RightLabel{\FalseInt}
\UnaryInfC{$!B$}
\DischargeRule{\Intro{\lif}}{1}
\UnaryInfC{$!A \lif !B$}
\DisplayProof
\]

Now suppose $\Gamma \Proves !B$.  Then there is !!a{derivation}~$\delta$ of
$!B$ from~$\Gamma$. The following !!{derivation} shows that $\Gamma
\Proves !A \lif !B$:
\[
\AxiomC{$\Discharge{!A}{1}$}
\AxiomC{}
\RightLabel{$\delta$}
\DeduceC{$!B$}
\RightLabel{\Intro{\land}}
\BinaryInfC{$!A \land !B$}
\RightLabel{\Elim{\land}}
\UnaryInfC{$!B$}
\DischargeRule{\Intro{\lif}}{1}
\UnaryInfC{$!A \lif !B$}
\DisplayProof
\]
(In fact we can simply add $\Intro{\lif}$ to~$\delta$, since
$\Intro{\lif}$ may, but does not have to, !!{discharge} the
assumption~$!A$.)}
\end{proof}

\begin{probtag}{probIf}
Prove \olref[fol][ntd][prv]{prop:provability-lif}.
\end{probtag}

\begin{thm}
\ollabel{thm:strong-generalization} If $c$ is a constant not occurring
in $\Gamma$ or $!A(x)$ and $\Gamma \Proves !A(c)$, then $\Gamma
\Proves \lforall[x][!A(x)]$.
\end{thm}

\begin{proof}
Let $\delta$ be !!a{derivation} of $!A(c)$ from $\Gamma$.  By adding a
\Intro{\lforall} inference, we obtain a proof of
$\lforall[x][!A(x)]$. Since $c$ does not occur in $\Gamma$ or $!A(x)$,
the eigenvariable condition is satisfied.
\end{proof}

\begin{thm}
\ollabel{thm:provability-quantifiers}
\begin{tagenumerate}{prvEx,prvAll}
\tagitem{prvEx}{If $\Gamma \Proves !A(t)$ then $\Gamma \Proves
  \lexists[x][!A(x)]$.}{}

\tagitem{prvAll}{If $\Gamma \Proves \lforall[x][!A(x)]$ then $\Gamma
  \Proves !A(t)$.}{}
\end{tagenumerate}
\end{thm}

\begin{proof}
\begin{tagenumerate}{prvEx,prvAll}
\tagitem{prvEx}{Suppose $\Gamma \Proves !A(t)$. Then there is
  !!a{derivation}~$\delta$ of $!A(t)$ from~$\Gamma$. The !!{derivation}
\[
\AxiomC{}
\RightLabel{$\delta$}
\DeduceC{$!A(t)$}
\RightLabel{\Intro{\lexists}}
\UnaryInfC{$\lexists[x][!A(x)]$}
\DisplayProof
\]
shows that $\Gamma \Proves \lexists[x][!A(x)]$.}{}

\tagitem{prvAll}{Suppose $\Gamma \Proves \lforall[x][!A(x)]$. Then there is
  !!a{derivation} $\delta$ of $\lforall[x][!A(x)]$ from~$\Gamma$.  The
  !!{derivation}
\[
\AxiomC{}
\RightLabel{$\delta$}
\DeduceC{$\lforall[x][!A(x)]$}
\RightLabel{\Elim{\lforall}}
\UnaryInfC{$\Atom{!A}{t}$}
\DisplayProof
\]
shows that $\Gamma \Proves !A(t)$.}{}

\end{tagenumerate}
\end{proof}
\end{document}
