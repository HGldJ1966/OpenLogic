% Part: first-order-logic
% Chapter: completeness
% Section: complete-sets

% Definition of complete consistent sets. Properties of complete sets required
% for completeness proved are in provability.tex in the
% chapter on the proof system used.

\documentclass[../../../include/open-logic-section]{subfiles}

\begin{document}

\olfileid{fol}{com}{ccs}
\olsection{Complete Consistent Sets of \usetoken{P}{sentence}}

\begin{defn}[Complete set]
\ollabel{def:complete-set} A set~$\Gamma$ of !!{sentence}s is
\emph{!!{complete}} iff for any !!{sentence}~$!A$, either $!A \in
\Gamma$ or $\lnot !A \in \Gamma$.
\end{defn}

\begin{explain}
!!^{complete} sets of sentences leave no questions unanswered. For
any !!{sentence}~$A$, $\Gamma$ ``says'' if $!A$ is true or false.  The
importance of !!{complete} sets extends beyond the proof of the
completeness theorem. A theory which is !!{complete} and
axiomatizable, for instance, is always decidable.
\end{explain}

\begin{explain}
!!^{complete} consistent sets are important in the completeness proof
since we can guarantee that every consistent set of
!!{sentence}s~$\Gamma$ is contained in a !!{complete} consistent
set~$\Gamma^*$.  !!^a{complete} consistent set contains, for each
!!{sentence}~$!A$, either $!A$ or its negation $\lnot !A$, but not
both. This is true in particular for atomic !!{sentence}s, so from
!!a{complete} consistent set in a language suitably expanded by
!!{constant}s, we can construct a !!{structure} where the
interpretation of !!{predicate}s is defined according to which atomic
!!{sentence}s are in~$\Gamma^*$. This !!{structure} can then be shown
to make all !!{sentence}s in~$\Gamma^*$ (and hence also all those
in~$\Gamma$) true. The proof of this latter fact requires that $\lnot
!A \in \Gamma^*$ iff $!A \notin \Gamma^*$, $(!A \lor !B) \in \Gamma^*$
iff $!A \in \Gamma^*$ or $!B \in \Gamma^*$, etc.
\end{explain}

In what follows, we will often tacitly use the properties of
reflexivity, monotonicity, and transitivity of $\Proves$ (see
\tagrefs{prfSC/{fol:seq:ptn:sec},prfND/{fol:ntd:ptn:sec}}).

\begin{prop}
\ollabel{prop:ccs}
Suppose $\Gamma$ is !!{complete} and consistent. Then:
\begin{enumerate}
\item \ollabel{prop:ccs-prov-in} If $\Gamma \Proves !A$, then $!A \in
  \Gamma$.

\tagitem{prvAnd}{\ollabel{prop:ccs-and} $!A \land !B \in \Gamma$
  iff both $!A \in \Gamma$ and $!B \in \Gamma$.}{}

\tagitem{prvOr}{\ollabel{prop:ccs-or} $!A \lor !B \in \Gamma$ iff
  either $!A \in \Gamma$ or $!B \in \Gamma$.}{}

\tagitem{prvIf}{\ollabel{prop:ccs-if} $!A \lif !B \in \Gamma$ iff
  either $!A \notin \Gamma$ or $!B \in \Gamma$.}{}
\end{enumerate}
\end{prop}

\begin{proof}
Let us suppose for all of the following that $\Gamma$ is !!{complete} and
consistent.
\begin{enumerate}
\item If $\Gamma \Proves !A$, then $!A \in \Gamma$.

Suppose that $\Gamma \Proves !A$. Suppose to the contrary that $!A
\notin \Gamma$. Since $\Gamma$ is !!{complete}, $\lnot !A \in \Gamma$.
By
\tagrefs{prfSC/{fol:seq:prv:prop:explicit-inc},prfND/{fol:ntd:prv:prop:explicit-inc}},
$\Gamma$ is inconsistent. This contradicts the assumption that
$\Gamma$ is consistent. Hence, it cannot be the case that $!A \notin
\Gamma$, so $!A \in \Gamma$.

\tagitem{defAnd}{}{%
\iftag{probAnd}{Exercise.}{%
$!A \land !B \in \Gamma$ iff both $!A \in \Gamma$ and $!B \in \Gamma$:

For the forward direction, suppose $!A \land !B \in \Gamma$. Then
by
\tagrefs{prfSC/{fol:seq:prv:prop:provability-land},prfND/{fol:ntd:prv:prop:provability-land}}, item~(1),
$\Gamma \Proves !A$ and $\Gamma \Proves !B$. By
\olref{prop:ccs-prov-in}, $!A \in \Gamma$ and $!B \in \Gamma$, as
required.

For the reverse direction, let $!A \in \Gamma$ and $!B \in
\Gamma$. By
\tagrefs{prfSC/{fol:seq:prv:prop:provability-land},prfND/{fol:ntd:prv:prop:provability-land}}, item~(2),
$\Gamma \Proves !A \land !B$. By \olref{prop:ccs-prov-in}, $!A \land
!B \in \Gamma$.}}

\tagitem{defOr}{}{%
\iftag{probOr}{Exercise.}{%
First we show that if $!A \lor !B \in \Gamma$, then either $!A \in
\Gamma$ or $!B \in \Gamma$. Suppose $!A \lor !B \in \Gamma$ but $!A
\notin \Gamma$ and $!B \notin \Gamma$.  Since $\Gamma$ is
!!{complete}, $\lnot !A \in \Gamma$ and $\lnot !B \in \Gamma$. By
\tagrefs{prfSC/{fol:seq:prv:prop:provability-lor},prfND/{fol:ntd:prv:prop:provability-lor}},
item (1), $\Gamma$ is inconsistent, a contradiction. Hence, either $!A
\in \Gamma$ or $!B \in \Gamma$.

For the reverse direction, suppose that $!A \in \Gamma$ or $!B \in
\Gamma$. By
\tagrefs{prfSC/{fol:seq:prv:prop:provability-lor},prfND/{fol:ntd:prv:prop:provability-lor}}, item (2),
$\Gamma \Proves !A \lor !B$. By \olref{prop:ccs-prov-in}, $!A \lor
!B \in \Gamma$, as required.}}

\tagitem{defIf}{}{%
\iftag{probIf}{Exercise.}{%
For the forward direction, suppose $!A \lif !B \in \Gamma$, and suppose
to the contrary that $!A \in \Gamma$ and $!B \notin \Gamma$. On these
assumptions, $!A \lif !B \in \Gamma$ and $!A \in \Gamma$. By
\tagrefs{prfSC/{fol:seq:prv:prop:provability-lif},prfND/{fol:ntd:prv:prop:provability-lif}}, item (1),
$\Gamma \Proves !B$. But then by \olref{prop:ccs-prov-in}, $!B \in
\Gamma$, contradicting the assumption that $!B \notin \Gamma$.

For the reverse direction, first consider the case where $!A \notin
\Gamma$. Since $\Gamma$ is !!{complete}, $\lnot !A \in \Gamma$.  By
\tagrefs{prfSC/{fol:seq:prv:prop:provability-lif},prfND/{fol:ntd:prv:prop:provability-lif}}, item (2),
$\Gamma \Proves !A \lif !B$. Again by \olref{prop:ccs-prov-in}, we get
that $!A \lif !B \in \Gamma$, as required.

Now consider the case where $!B \in \Gamma$.  By
\tagrefs{prfSC/{fol:seq:prv:prop:provability-lif},prfND/{fol:ntd:prv:prop:provability-lif}},
item (2) again, $\Gamma \Proves !A \lif !B$. By
\olref{prop:ccs-prov-in}, $!A \lif !B \in \Gamma$.}}
\end{enumerate}
\end{proof}

\begin{prob}
Complete the proof of \olref[fol][com][ccs]{prop:ccs}.
\end{prob}

\end{document}
