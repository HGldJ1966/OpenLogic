% Part: incompleteness
% Chapter: incompleteness-provability
% Section: provability-conditions

\documentclass[../../../include/open-logic-section]{subfiles}

\begin{document}

\olfileid{inc}{inp}{prc}

\olsection{The Provability Conditions for $\Th{PA}$}

Peano arithmetic, or $\Th{PA}$, is the theory extending $\Th{Q}$ with
induction axioms for all !!{formula}s. In other words, one adds to $\Th{Q}$
axioms of the form
\[
(!A(0) \land \lforall[x][(!A(x) \lif !A(x'))]) \lif \lforall[x][!A(x)]
\]
for every !!{formula}~$!A$. Notice that this is really a {\em schema},
which is to say, infinitely many axioms (and it turns out that
$\Th{PA}$ is {\em not} finitely axiomatizable). But since one can
effectively determine whether or not a string of symbols is an
instance of an induction axiom, the set of axioms for $\Th{PA}$ is
computable. $\Th{PA}$ is a much more robust theory than $\Th{Q}$. For
example, one can easily prove that addition and multiplication are
commutative, using induction in the usual way. In fact, most finitary
number-theoretic and combinatorial arguments can be carried out
in~$\Th{PA}$.

Since $\Th{PA}$ is computably axiomatized, the provability predicate
$\Prf[\Th{PA}](x,y)$ is computable and hence represented in~$\Th{Q}$ (and
so, in~$\Th{PA}$). As before, I will take $\OPrf[\Th{PA}](x,y)$ to denote
the formula representing the relation.  Let $\OProv[\Th{PA}](y)$ be the
formula $\lexists[x][\Prf[\Th{PA}](x,y)]$, which, intuitively says, ``$y$ is
provable from the axioms of $\Th{PA}$.''  The reason we need a little
bit more than the axioms of $\Th{Q}$ is we need to know that the
theory we are using is strong enough to prove a few basic facts about
this provability predicate. In fact, what we need are the following
facts:
\begin{enumerate}
\item[P1.] If $\Th{PA} \Proves !A$, then $\Th{PA} \Proves
  \OProv[\Th{PA}](\gn{!A})$
\item[P2.] For all !!{formula}s $!A$ and $!B$,
  \[
  \Th{PA} \Proves \OProv[\Th{PA}](\gn{!A \lif !B}) \lif
  (\OProv[\Th{PA}](\gn{!A}) \lif \OProv[\Th{PA}](\gn{!B}))
  \]
\item[P3.] For every !!{formula}~$!A$,
  \[
  \Th{PA} \Proves \OProv[\Th{PA}](\gn{!A})
  \lif \OProv[\Th{PA}](\gn{\OProv[\Th{PA}](\gn{!A})}).
  \]
\end{enumerate}
The only way to verify that these three properties hold is to describe
the !!{formula} $\OProv[\Th{PA}](y)$ carefully and use the axioms of
$\Th{PA}$ to describe the relevant formal proofs. Conditions (1)
and~(2) are easy; it is really condition~(3) that requires
work. (Think about what kind of work it entails\dots) Carrying out the
details would be tedious and uninteresting, so here we will ask you to
take it on faith that $\Th{PA}$ has the three properties listed
above. A reasonable choice of $\OProv[\Th{PA}](y)$ will also satisfy
\begin{enumerate}
\item[P4.] If $\Th{PA} \Proves \OProv[\Th{PA}](\gn{!A})$, then
  $\Th{PA} \Proves !A$.
\end{enumerate}
But we will not need this fact.

\begin{digress}
Incidentally, G\"odel was lazy in the same way we are
being now. At the end of the 1931 paper, he sketches the proof of the
second incompleteness theorem, and promises the details in a later
paper. He never got around to it; since everyone who understood the
argument believed that it could be carried out (he did not need to
fill in the details.)
\end{digress}

\end{document}
