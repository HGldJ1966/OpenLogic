% Part: incompleteness
% Chapter: introduction
% Section: decidability

\documentclass[../../../include/open-logic-section]{subfiles}

\begin{document}

\olfileid{inc}{int}{dec}

\olsection{Undecidability and Incompleteness}

G\"odel's proof of the incompleteness theorems require arithmetization
of syntax. But even without that we can obtain some nice results just
on the assumtion that a theory !!{represents} all !!{decidable}
relations.  The proof is a diagonal argument similar to the proof of
the undecidability of the halting problem.

\begin{thm}
If $\Gamma$ is a consistent theory that !!{represents} every
!!{decidable} relation, then $\Gamma$ is not !!{decidable}.
\end{thm}

\begin{proof}
Suppose $\Gamma$ were !!{decidable}. We show that if $\Gamma$
!!{represents} every !!{decidable} relation, it must be inconsistent.

!!^{decidable} properties (one-place relations) are represented by
!!{formula}s with one free variable. Let $!A_0(x)$, $!A_1(x)$, \dots,
be a computable enumeration of all such !!{formula}s.  Now consider
the following set $D \subseteq \Nat$:
\[
D = \Setabs{n}{\Gamma \Proves \lnot !A_n(\num{n})}
\]
The set $D$ is !!{decidable}, since we can test if $n \in D$ by first
computing $!A_n(x)$, and from this $\lnot !A_n(\num{n})$. Obviously,
substituting the term $\num{n}$ for every free occurrence of $x$ in
$!A_n(x)$ and prefixing $!A(\num{n})$ by $\lnot$ is a mechanical
matter.  By assumption, $\Gamma$ is !!{decidable}, so we can test if
$\lnot !A(\num{n}) \in \Gamma$. If it is, $n \in D$, and if it isn't,
$n \notin D$. So $D$ is likewise !!{decidable}.

Since $\Gamma$ !!{represents} all !!{decidable} properties, it
!!{represents}~$D$.  And the !!{formula}s which !!{represents}s $D$ in
$\Gamma$ are all among $!A_0(x)$, $!A_1(x)$, \dots. So let $d$ be a
number such that $!A_d(x)$ !!{represents} $D$ in~$\Gamma$.  If $d
\notin D$, then, since $!A_d(x)$ !!{represents}~$D$, $\Gamma \Proves
\lnot !A_d(\num{d})$. But that means that $d$ meets the defining
condition of~$D$, and so $d \in D$. This contradicts $d \notin D$. So
by indirect proof, $d \in D$.

Since $d \in D$, by the definition of~$D$, $\Gamma \Proves \lnot
!A_d(\num{d})$. On the other hand, since $!A_d(x)$ !!{represents}~$D$
in $\Gamma$, $\Gamma \Proves !A_d(\num{d})$. Hence, $\Gamma$ is
inconsistent.
\end{proof}

\begin{explain}
The preceding theorem shows that no theory that !!{represents} all
!!{decidable} relations can be !!{decidable}. We will show that
$\Th{Q}$ does !!{represents}s all !!{decidable} relations; this means
that all theories that include $\Th{Q}$, such as $\Th{PA}$ and
$\Th{TA}$, also do, and hence also are not !!{decidable}.

We can also use this result to obtain a weak version of the first
incompleteness theorem.  Any theory that is !!{axiomatizable} and
!!{complete} is !!{decidable}.  Consistent theories that are
!!{axiomatizable} and !!{represents}s all !!{decidable} properties
then cannot be !!{complete}.
\end{explain}

\begin{thm}
If $\Gamma$ is !!{axiomatizable} and !!{complete} it is !!{decidable}.
\end{thm}

\begin{proof}
Any inconsistent theory is !!{decidable}, since inconsistent theories
contain all !!{sentence}s, so the answer to the question ``is $!A \in
\Gamma$'' is always ``yes,'' i.e., can be decided.

So suppose $\Gamma$ is consistent, and furthermore is
!!{axiomatizable}, and !!{complete}. Since $\Gamma$ is
!!{axiomatizable}, it is !!{computably enumerable}. For we can
enumerate all the correct !!{derivation}s from the axioms of~$\Gamma$
by a computable function. From a correct !!{derivation} we can compute
the !!{sentence} it !!{derive}s, and so together there is a computable
function that enumerates all theorems of~$\Gamma$.  A !!{sentence} is
a theorem of~$\Gamma$ iff $\lnot !A$ is not a theorem, since $\Gamma$
is consistent and !!{complete}.  We can therefore decide if $!A \in
\Gamma$ as follows. Enumerate all theorems of $\Gamma$. When $!A$
appears on this list, we know that $\Gamma \Proves !A$. When $\lnot
!A$ appears on this list, we know that $\Gamma \Proves/ !A$.  Since
$\Gamma$ is !!{complete}, one of these cases eventually obtains, so
the procedure eventually produces and answer.
\end{proof}

\begin{cor}
\ollabel{cor:incompleteness}
If $\Gamma$ is consistent, !!{axiomatizable}, and !!{represents} every
!!{decidable} property, it is not !!{complete}.
\end{cor}

\begin{proof}
If $\Gamma$ were !!{complete}, it would be !!{decidable} by the
previous theorem (since it is !!{axiomatizable} and consistent). But
since $\Gamma$ !!{represents} every !!{decidable} property, it is not
!!{decidable}, by the first theorem.
\end{proof}

\begin{prob}
Show that $\Th{TA} = \Setabs{!A}{\Sat{N}{!A}}$ is not
!!{axiomatizable}. You may assume that $\Th{TA}$ represents all
decidable properties.
\end{prob}

Once we have established that, e.g., $\Th{Q}$, !!{represents} all
!!{decidable} properties, the corollary tells us that $\Th{Q}$ must be
incomplete. However, its proof does not provide an example of an
independent !!{sentence}; it merely shows that such a !!{sentence}
must exist. For this, we have to arithmetize syntax and follow
G\"odel's original proof idea.  And of course, we still have to show
the first claim, namely that $\Th{Q}$ does, in fact, !!{represents}s
all !!{decidable} properties.

\end{document}
